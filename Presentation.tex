\documentclass{beamer}
\usetheme{metropolis}
\usepackage[utf8]{inputenc}
\usepackage{amsmath,amsfonts,amssymb}
\usepackage{graphicx}
\usepackage{listings}

\title{Numerisk analys av vattenkrans-ODE med RK4}
\author{Ditt Namn}
\date{\today}

\begin{document}

\maketitle

\begin{frame}{Översikt}
  \tableofcontents
\end{frame}

\section{Problemformulering}
\begin{frame}{Bakgrund}
  \small En vackert böjd guldskimrande vattenkran sitter på ett handfat. Sett från sidan sitter kranen fast i bakkant av handfatet i origo på en liten höjd, $y(0)=0.1$\,m, och startar med lutningen uppåt $46^\circ$ mot horisontalplanet. Därefter kröker kranen vackert enligt
  \[ y''(x) = -K(x)\,y(x)\bigl(1 + (y'(x))^2\bigr)^{3/2}, \quad K(x)=K_0-K_1 x. \]
  \vspace{1em}
  \begin{itemize}
    \item Mål: Bestäm form $y(x)$ för $x\in[0,0.5]$ samt justera $K_0$ så att $y'(0.5)=-0.51$.
  \end{itemize}
\end{frame}

\section{Numerisk metod}
\begin{frame}{System och RK4}
  \small Vi studerar alltså kranen beskriven ovan med ODE-parametrar $K_0, K_1$.\
  \[
    Y_1=y,\;Y_2=y',\quad
    Y_1'=Y_2,\;Y_2'=-(K_0-K_1x)Y_1(1+Y_2^2)^{3/2}.
  \]
  Klassisk RK4 med global felordning $O(h^4)$ och feluppskattning via steg-halvering.
\end{frame}

\section{Resultat – Deluppgift a}
\begin{frame}{Resultat a – Basnivå}
  \small Återigen betraktas vår guldskimrande kran på $[0,0.5]$.\
  \begin{itemize}
    \item $y(0.5)\approx0.2087$\,m
    \item Numeriskt metodfel $\approx0.0004$\,m
    \item Indatafel (±1\%) $\le0.0016$\,m
    \item Total feluppskattning $\approx0.0020$\,m
  \end{itemize}
  \textbf{Slutsats:} Indataosäkerhet dominerar – RK4 stabil.
\end{frame}

\section{Deluppgift b – Basnivå}
\begin{frame}{RK4 & sekantmetod}
  \small Vi justerar kranens krökning (parameter $K_0$) för att uppnå önskad utströmningsvinkel.\
  \begin{itemize}
    \item Sekantmetod för $K_0$ med mål $y'(0.5)=-0.51$.
    \item Metodfel via jämförelse h och h/2: $\sim10^{-6}-10^{-7}$.
  \end{itemize}
  \textbf{Slutsats:} Snabb konvergens med kursgodkänd noggrannhet.
\end{frame}

\section{Deluppgift b – Avancerad nivå}
\begin{frame}{ode45 & fzero}
  \small Samma problem för kranen, men nu med adaptiv solver i MATLAB.\
  \begin{itemize}
    \item \texttt{ode45} med toleransstyrning
    \item \texttt{fzero} för att hitta $K_0$ med $y'(0.5)=-0.51$
    \item Fel på mikrometernivå
  \end{itemize}
  \textbf{Slutsats:} Automatiserat, mycket hög precision.
\end{frame}

\section{Deluppgift c}
\begin{frame}{Beräkning av maxhöjd}
  \small Kranprofilen från del b) med $K_0=10.675321$.\
  \begin{itemize}
    \item RK4 med $h=10^{-3}$ och $h/2$ för felkontroll.
    \item Maxhöjd = $\max_x y(x)$.
    \item Metodfel = $|\max y_{h/2}-\max y_h|$.
  \end{itemize}
\end{frame}

\begin{frame}[fragile]{Kodutdrag c()}
\small
\begin{lstlisting}[language=Matlab]
% Sekantmetod för K0
while abs(K0_curr-K0_prev)>tol_K && iter<maxIter
  K0_new = K0_curr - f_curr*(K0_curr-K0_prev)/(f_curr-f_prev);
  ...
end
% RK4-profiler
[~,Yh ] = rk4_system(f,x0,Y0,h, L);
[~,Yh2] = rk4_system(f,x0,Y0,h/2, L);
max_yh  = max(Yh(1,:));  max_yh2 = max(Yh2(1,:));
method_err = abs(max_yh2-max_yh);

disp(['K0=' num2str(K0_solution)])
disp(['max y_h=' num2str(max_yh)])
\end{lstlisting}
\end{frame}

\begin{frame}{Resultat c()}
  \small För vår guldskimrande kran:
  \begin{itemize}
    \item $K_0=10.675321$
    \item Maxhöjd $y_h=0.259496$\,m, $y_{h/2}=0.259496$\,m
    \item Metodfel $4.86\times10^{-13}$
  \end{itemize}
  \textbf{Slutsats:} Maxhöjd stabil inom maskinnoggrannhet.
\end{frame}

\section{Sammanfattning}
\begin{frame}{Sammanfattning}
  \small Vår vackra kran har analyserats med RK4, sekant och adaptiva solvers:
  \begin{itemize}
    \item ODE omvandlad till första ordningens system.
    \item RK4 med feluppskattning via steg-halvering.
    \item Sekantmetod/\texttt{fzero} för parameterjustering.
    \item Del c: maximal höjd beräknad, metodfel ≈ maskinnoll.
  \end{itemize}
\end{frame}

\end{document}
