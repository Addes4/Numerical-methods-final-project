\documentclass[12pt, letterpaper]{article}
\usepackage{graphicx}
\usepackage{amsmath}
\usepackage{amssymb}
\usepackage{physics}
\usepackage{siunitx} % For \degree

\begin{document}


\title{Numeriska Metoder SF1546}
\author{Laboranter: Axel Öberg, Adrian Sohrabi \\ \small Projekt B på en högre nivå.}
\date{VT 2025}

\maketitle
\newpage
\section{Introduktion}
Detta dokument beskriver den matematiska teorin bakom lösningen av ett system av ordinära differentialekvationer (ODE) med hjälp av Runge-Kutta av fjärde ordningen (RK4). Specifikt behandlas ekvationen:
\begin{equation}
y'' = - (K_0 - K_1 x) y (1 + (y')^2)^{3/2},
\end{equation}
vilken är en andra ordningens differentialekvation.

\section{Problemformulering}
En vackert böjd, guldskimrande vattenkran sitter på ett handfat. Sett från sidan sitter kranen fast i bakkant av handfatet i origo på en liten höjd, vilket beskrivs av initialvillkoret:
\begin{equation}
y(0) = 0.1.
\end{equation}\\
Kranens lutning vid starten är uppåtriktad med en vinkel på \SI{46}{\degree} mot horisontalplanet, vilket innebär att:
\begin{equation}
y'(0) = \tan(\SI{46}{\degree}).
\end{equation}\\
Kranens form beskrivs av följande differentialekvationssystem:
\begin{equation}
y'' = -K(x)y(x)(1 + (y'(x))^2)^{3/2},
\end{equation}
där funktionen $K(x)$ ges av:
\begin{equation}
K(x) = K_0 - K_1 x.
\end{equation}\\
Här betecknar $y'$ och $y''$ första- respektive andraderivatan med avseende på variabeln $x$, som mäter avståndet i sidled när kranen ses från sidan.

\section{Omskrivning till Första Ordningens System}
För att kunna lösa ekvationen numeriskt med RK4, omskrivs den till ett system av första ordningens differentialekvationer genom substitution:
\begin{equation}
\begin{cases}
y_1' = y_2, \\
y_2' = - (K_0 - K_1 x) y_1 (1 + y_2^2)^{3/2}.
\end{cases}
\end{equation}
Detta system skrivs ofta i matrisform för bättre översikt:
\begin{equation}
\begin{bmatrix}
y_1' \\
y_2'
\end{bmatrix} =
\begin{bmatrix}
y_2 \\
- (K_0 - K_1 x) y_1 (1 + y_2^2)^{3/2}
\end{bmatrix}.
\end{equation}
Detta system kan nu lösas numeriskt.

\section{Numerisk Justering av $K_0$ med Sekantmetoden}
För att hitta ett lämpligt värde på $K_0$ som uppfyller villkoret $y'(0.5) = -0.51$, används sekantmetoden. Sekantmetoden är en iterativ rotlösningsmetod som bygger på följande rekursion:
\begin{equation}
K_0^{(n+1)} = K_0^{(n)} - f(K_0^{(n)}) \frac{K_0^{(n)} - K_0^{(n-1)}}{f(K_0^{(n)}) - f(K_0^{(n-1)}}.
\end{equation}
Där $f(K_0) = y'(0.5) + 0.51$ representerar felet i derivatans slutvärde.

Sekantmetoden upprepas tills skillnaden mellan två på varandra följande gissningar för $K_0$ är mindre än en toleransnivå $\varepsilon$ eller tills det maximala antalet iterationer uppnåtts.

\section{Newtons Metod och Finita Differenser}
F"or att justera parametern $K_0$ och startlutningen $s$ anv"ander vi Newtons metod. Vi definierar residualerna:
\begin{align}
    R_1 &= y'(0.5) - (-0.51), \\
    R_2 &= \max(y) - 0.255.
\end{align}
Dessa uttryck beskriver skillnaden mellan den numeriska l"osningen och de "onskade villkoren.

Vi approximerar Jacobianen $J$ numeriskt med finita differenser:
\begin{equation}
    J_{ij} \approx \frac{F_i(p + \epsilon e_j) - F_i(p)}{\epsilon},
\end{equation}
d"ar $e_j$ "ar en enhetsvektor och $\epsilon$ "ar ett litet stegv"arde.

Newton-uppdateringen ges av:
\begin{equation}
    p^{(k+1)} = p^{(k)} - J^{-1} F(p^{(k)}).
\end{equation}
Vi itererar tills $\|F(p)\|$ "ar mindre "an en given tolerans.

\section{Interpolering av Maximum}
F"or att approximera det maximala v"ardet av $y(x)$ v"aljer vi tre punkter kring maximum och anpassar en andragradspolynom:
\begin{equation}
    y = ax^2 + bx + c.
\end{equation}
Genom att l"osa ekvationssystemet
\begin{equation}
    \begin{bmatrix}
        x_1^2 & x_1 & 1 \\
        x_2^2 & x_2 & 1 \\
        x_3^2 & x_3 & 1
    \end{bmatrix}
    \begin{bmatrix} a \\ b \\ c \end{bmatrix} =
    \begin{bmatrix} y_1 \\ y_2 \\ y_3 \end{bmatrix},
\end{equation}
finner vi koefficienterna $a$, $b$, och $c$. Maximum ges sedan av:
\begin{equation}
    x_{\text{max}} = -\frac{b}{2a}, \quad y_{\text{max}} = a x_{\text{max}}^2 + b x_{\text{max}} + c.
\end{equation}


\begin{document}

\section*{Differentialekvationssystem}

Det givna systemet av differentialekvationer kan skrivas som:

\section*{Differentialekvationssystem}

Vi betraktar differentialekvationen
\begin{equation}
    \frac{dY}{dx} = \begin{bmatrix} y' \\ - (K_0 - K_1 x) y \left(1 + (y')^2\right)^{3/2} \end{bmatrix},
\end{equation}
där vi genom substitutionen
\[
y = Y_1 \quad \text{och} \quad y' = Y_2 = v
\]
omvandlar ekvationen till ett system av två första ordningens differentialekvationer:
\begin{align}
    y' &= v, \\
    v' &= - (K_0 - K_1 x) y \left(1 + v^2\right)^{3/2}.
\end{align}

\end{document}

